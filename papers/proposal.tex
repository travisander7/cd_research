\documentclass{article}
\usepackage{amsmath}

\title{Creating an exact confidence distribution for rare event meta-analyses}
\author{Travis Andersen}

\begin{document}
\maketitle  

When performing statistical inference on a parameter, it is common to compute p-values and confidence intervals to evaluate a certain hypothesis. When doing so, a level of significance is set which determines what level of evidence is needed to reject the null hypothesis. The conclusion generated by this approach depends on the level of significance set at the start of the study. Confidence distributions are constructed by creating confidence intervals for all levels of significance. They contain confidence intervals, p-values, and more information on the parameter while eliminating the subjective choice of one level of significance. 

In the medical field, it is very common to perform meta-analyses of rare events. Traditional statistical methods, which rely on an asymptotic normal approximation, often perform poorly in this situation. Dr. Zabriskie has developed an exact permutation based approach that performs well in rare event settings. Its main advantage is the preservation of the nominal level of significance. 

For my thesis, I will help develop a confidence distribution for the exact permutation method developed by Dr. Zabriskie. This work will enable researchers to view the exact permutation method results at any level of significance. We will develop a mathematical proof that our proposed method is a confidence distribution, and we will also perform a simulation study to evaluate its properties. Additionally, we will develop new ways to summarize the overall performance of a confidence distribution. We plan to submit our work to \emph{Statistics in Medicine}.

\end{document}