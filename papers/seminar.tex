\documentclass{beamer}

\title{A Permutation-Based Confidence Distribution for Rare-Event Meta-Analysis}
\author{Travis Andersen \and Dr Brinley Zabriskie}

\begin{document}

\begin{frame}
  \maketitle
\end{frame}

% Multiple studies often occur in medical field
% Beta-blockers prevent heart disease
% The parameter of interest is the log odds ratio
\begin{frame} 
  \frametitle{Meta-analysis}
  \begin{itemize}
    \item Meta-analyses are used to combine information from independent studies.
    \item Zabriskie et al. developed an exact permutation method for rare event binary meta-analyses.
  \end{itemize}
  \begin{table}
    \small
    \begin{tabular}{|l|r|r|r|r|}
      \hline
      Study & Trt. Event & Trt. Total & Ctrl. Event & Ctrl. Total\\
      \hline
      Barber & 10 & 52 & 12 & 47\\
      \hline
      Norris & 5 & 226 & 15 & 228\\
      \hline
      Kahler & 0 & 38 & 6 & 31\\
      \hline
      Ledwich & 2 & 20 & 3 & 20\\
      \hline
    \end{tabular}
  \end{table}
\end{frame}

\begin{frame} 
  \frametitle{Confidence distributions}
  \begin{itemize}
    \item Most meta-analysis methods produce a p-value and confidence interval at a single level of significance.
    \item Confidence distributions (CDs) provide a comprehensive overview of the available inference on a parameter at all levels of significance.
  \end{itemize}
\end{frame}

\begin{frame} 
  \frametitle{Confidence curves}
  \begin{itemize}
    \item A confidence curve (CV) is a function of a CD that is used to easily
    visualize CIs and extract p-values.
  \end{itemize}
  \begin{figure}
    \includegraphics[scale=0.4]{cv.png}
  \end{figure}
\end{frame}

\begin{frame}
  \frametitle{Confidence curves in meta-analysis}
  \begin{itemize}
    \item Some meta-analysis methods combine individual CVs to produce one overall CV.
  \end{itemize}
  \begin{figure}
    \includegraphics[scale=0.4]{meta.png}
  \end{figure}
\end{frame}

\begin{frame}
  \frametitle{Research goal}
  \begin{itemize}
    \item The method developed by Zabriskie et al. requires choosing a single level of significance.
    \item We want to utilize CDs for this method.
  \end{itemize}
\end{frame}

\begin{frame} 
  \frametitle{Definition of CD}
   Let $(\theta_-, \infty)$ be a one-sided $100(1-\alpha)\%$ CI for $\theta$, where the lower confidence bound $\theta_-$ is a function of the level of significance, $\alpha$. If, for every $\alpha \in (0, 1)$ and $\theta \in \Theta$, $\theta_-(\alpha)$ is continuous and increasing in $\alpha$ for each sample $\textbf{z}$, then $\theta_{-}^{-1}(\cdot) = \phi(\theta)$ is a CD.
\end{frame}


\begin{frame}
  \frametitle{Exact permutation CD}
  \begin{equation*} \label{eq:permCI}
    R(\theta) = \frac
    {\sum_{u^* = t(\mathbf{z}_{\text{obs}})}^{\max(t(\mathbf{z}))} \text{C}(u^* \vert \cdot) \exp\{\theta u^*\}}
    {\sum_{u = \min(t(\mathbf{z}))}^{\max(t(\mathbf{z}))} \text{C}(u \vert \cdot) \exp\{\theta u\}} \text{,}
  \end{equation*}
  One-sided $100(1-\alpha)$\% CIs can be given by $(R^{-1}(\alpha),\infty)$ and $(-\infty, R^{-1}(1-\alpha))$.

  A CD can be derived by inverting the lower confidence bound of a one-sided CI. Therefore, $\phi(\theta) = (R^{-1}(\alpha))^{-1} = R(\theta)$. 
\end{frame}

\begin{frame} 
  \frametitle{Evaluating CVs}
  \begin{itemize}
    \item When evaluating CIs, coverage and width are calculated.
    \item We plan on using area under the curve and height at the true parameter value as their CV counterparts. 
  \end{itemize}
  \begin{figure}
    \includegraphics[scale=0.35]{cv.png}
  \end{figure}
\end{frame}

\begin{frame} 
  \frametitle{Conclusion}
  \begin{itemize}
    \item We calculated the CD for the method developed by Zabriskie et al. 
    \item We found ways to evaluate the performance of a CV.
  \end{itemize}
\end{frame}

\end{document}