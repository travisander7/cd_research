\documentclass{article}
\usepackage{amsmath}

\title{Proof for CD}
\author{Travis Andersen}

\begin{document}
\maketitle  

Goal: Show lower confidence bound $\theta_- = R^{-1}(\alpha)$ is continuous and increasing in $\alpha$. 

\begin{equation} \label{eq:permCI}
  R(\theta) = \frac
  {\sum_{u^* = t(\mathbf{z}_{\text{obs}})}^{\max(t(\mathbf{z}))} \text{C}(u^* \vert \cdot) \exp\{\theta u^*\}}
  {\sum_{u = \min(t(\mathbf{z}))}^{\max(t(\mathbf{z}))} \text{C}(u \vert \cdot) \exp\{\theta u\}}
\end{equation}

\begin{itemize}
  \item If a function has an inverse, it is bijective.
  \item If a bijective function is increasing, then its inverse is increasing
  \item Therefore, we only need show that $R(\theta)$ is increasing to show that $R^{-1}(\alpha)$ is increasing. 
\end{itemize}

Following are a few ideas I have had on showing that $R(\theta)$ is increasing in $\theta$. 

\section{Easy method}

I didn't think this would work, but wanted to double check. Is it valid to write $R(\theta) = \sum_{u^* = t(\mathbf{z}_{\text{obs}})}^{\max(t(\mathbf{z}))} \text{C}^*(u^* \vert \cdot) \exp\{\theta u^*\}$, where $C^*(u^* \vert \cdot)$ is normalized? This would make showing that $R(\theta)$ is increasing very easy.

\section{Taking the derivative of $R(\theta)$}

We want to show that $\frac{\partial}{\partial \theta} \text{log}(R(\theta)) > 0$.

\begin{align*}
  \text{log}(R(\theta)) &= 
  \text{log}\left(\sum_{u^*=t(z_{obs})}^{\text{max}(t(z))} C(u^*|\cdot)e^{\theta u^*} \right) - 
  \text{log}\left(\sum_{u=\text{min}(t(z))}^{\text{max}(t(z))} C(u|\cdot)e^{\theta u}\right) \\
  \frac{\partial}{\partial \theta} \text{log}(R(\theta)) &= 
  \frac
  {\sum_{u^*=t(z_{obs})}^{\text{max}(t(z))} u^* C(u^*|\cdot)e^{\theta u^*}}
  {\sum_{u^*=t(z_{obs})}^{\text{max}(t(z))} C(u^*|\cdot)e^{\theta u^*}} -
  \frac
  {\sum_{u=\text{min}(t(z))}^{\text{max}(t(z))} u C(u|\cdot)e^{\theta u}}
  {\sum_{u=\text{min}(t(z))}^{\text{max}(t(z))} C(u|\cdot)e^{\theta u}} \\
  &\propto \left(\sum_{u^*=t(z_{obs})}^{\text{max}(t(z))} u^* C(u^*|\cdot)e^{\theta u^*}\right)
  \left(\sum_{u=\text{min}(t(z))}^{\text{max}(t(z))} C(u|\cdot)e^{\theta u}\right) - \\
  &\left(\sum_{u^*=t(z_{obs})}^{\text{max}(t(z))} C(u^*|\cdot)e^{\theta u^*}\right)
  \left(\sum_{u=\text{min}(t(z))}^{\text{max}(t(z))} u C(u|\cdot)e^{\theta u}\right) \\
  &\propto \left(\sum_{u^*=t(z_{obs})}^{\text{max}(t(z))} u^* C(u^*|\cdot)e^{\theta u^*}\right)
  \left(\sum_{u=\text{min}(t(z))}^{t(z_{obs})-1} C(u|\cdot)e^{\theta u} + \sum_{u=t(z_{obs})}^{\text{max}(t(z))} C(u|\cdot)e^{\theta u}\right) - \\
  &\left(\sum_{u^*=t(z_{obs})}^{\text{max}(t(z))} C(u^*|\cdot)e^{\theta u^*}\right)
  \left(\sum_{u=\text{min}(t(z))}^{t(z_{obs})-1} u C(u|\cdot)e^{\theta u} + \sum_{u=t(z_{obs})}^{\text{max}(t(z))} u C(u|\cdot)e^{\theta u}\right) \\
  &\propto \left(\sum_{u^*=t(z_{obs})}^{\text{max}(t(z))} u^* C(u^*|\cdot)e^{\theta u^*}\right)
  \left(\sum_{u=\text{min}(t(z))}^{t(z_{obs})-1} C(u|\cdot)e^{\theta u}\right) - \\
  &\left(\sum_{u^*=t(z_{obs})}^{\text{max}(t(z))} C(u^*|\cdot)e^{\theta u^*}\right)
  \left(\sum_{u=\text{min}(t(z))}^{t(z_{obs})-1} u C(u|\cdot)e^{\theta u}\right) \\
\end{align*}

I am not sure where to go from here. 

\section{Working with $R(\theta)$ directly}

We can write 
\begin{align*}
  R(\theta) &= \frac
  {\sum_{u^* = t(\mathbf{z}_{\text{obs}})}^{\max(t(\mathbf{z}))} \text{C}(u^* \vert \cdot) \exp\{\theta u^*\}}
  {\sum_{u = \min(t(\mathbf{z}))}^{\max(t(\mathbf{z}))} \text{C}(u \vert \cdot) \exp\{\theta u\}} \\
  &= \frac
  {\sum_{u^* = t(\mathbf{z}_{\text{obs}})}^{\max(t(\mathbf{z}))} \text{C}(u^* \vert \cdot) \exp\{\theta u^*\}}
  {\sum_{u = \min(t(\mathbf{z}))}^{t(\mathbf{z}_{\text{obs}})-1} \text{C}(u \vert \cdot) \exp\{\theta u\} + 
  \sum_{u^* = t(\mathbf{z}_{\text{obs}})}^{\max(t(\mathbf{z}))} \text{C}(u^* \vert \cdot) \exp\{\theta u^*\}} \\
  &= \frac{f(\theta)}{g(\theta) + f(\theta)}, \text{ where } \\
  f(\theta) &= \sum_{u^* = t(\mathbf{z}_{\text{obs}})}^{\max(t(\mathbf{z}))} \text{C}(u^* \vert \cdot) \exp\{\theta u^*\} \text{ and } \\
  g(\theta) &= \sum_{u = \min(t(\mathbf{z}))}^{t(\mathbf{z}_{\text{obs}})-1} \text{C}(u \vert \cdot) \exp\{\theta u\} \\
\end{align*}

If we can show $f(\theta)$ is increasing faster in $\theta$ than $g(\theta)$, that would indicate that $R(\theta)$ is increasing in $\theta$, right? Or maybe there is something else we can extract here. 

\end{document}